\documentclass[fleqn,10pt]{wlscirep}
\usepackage[utf8]{inputenc}
\usepackage[T1]{fontenc}
\begin{document}
\title{Understanding Self and Others Decal- Spring 2019}

\author{Natasha Batra, Sophie Yoo, Michelle Vo}

\thispagestyle{empty}
\maketitle
Mental illnesses affect 19 percent of the adult population, 46 percent of teenagers and 13 percent of children each year. People struggling with their mental health may be in your family, live next door, teach your children, work in the next cubicle or sit in the same church pew. However, only half of those affected receive treatment, often because of the stigma attached to mental health. Untreated, mental illness can contribute to higher medical expenses, poorer performance at school and work, fewer employment opportunities and increased risk of suicide.

In the decal, Understanding Self and Others, we strive to promote self-awareness and the discussion of mental health because unfortunately, the topic of mental health is undervalued and under-appreciated in today’s society. This 2-unit Pass/No Pass Decal gives students the opportunity to express their feelings and thoughts in a comfortable setting. While the Decal is informative in the sense that we inform students about the causes and symptoms of some mental health conditions, we also provide a place where students can get to know their fellow peers in a personal level by sharing similar experiences with each other. 

    \sectionmark  
    ***Note**: Any personal, sensitive experiences and stories that are shared during section meeting must STAY within the parameters of the Decal itself and no stories must be shared outside of the Decal in order to remain confidentiality and respectfulness out of the students who are discussing their experiences.
\section*{Information}

\begin{itemize}
    \item Units: 2, P/NP
    \item Section Times: Friday, 6:30-8pm 
    \item Facilitators: Natasha Batra, Sophie Yoo, Michelle Vo
\end{itemize}

\subsection*{Attendance}
Attendance is mandatory. Since the decal provides a space for students to talk about the sensitive topics of mental health, the purpose of the decal is for students to give support to their peers experiencing the same or different stressors of being a college student in UC Berkeley. Therefore, your weekly involvement and participation is necessary in order for this decal to function! 

\section*{Prerequisites}
No prerequisites necessary! Be prepared to openly share your personal experiences with mental health. This decal is designed to be a comforting environment where students are given the opportunity to talk about the topic of mental health. 

\section*{Assignments}
There will be mandatory in-paper reflections that will be due in class every Friday. The reflection must be at least one page. The reflections are intended for the students to reflect on what they have learned for the week, how they are feeling mentally and emotionally, and the steps that they have taken to improve on their stress, anxiety, or other mental health problems. Reflections will be worth 10 points each and will be graded on completion.

\section*{End of the Semester Project}
At the end of the semester, students are required to submit a 10 slide PowerPoint presentation where students will research and present information about a mental illness condition. Students must present in-class and are given the option to partner up with another student in the class. Students can choose any mental illness condition that they want to research on, but be sure to inform the facilitators ahead of time so that there would not be any repetition of topics during the day of presentations. The project will be worth 100 points and will be graded on thoroughness of how well the student presents his/her topic. 

\section*{Grading}
\begin{itemize}
    \item Attendance: 15/100
    \item Weekly Reflections: 20/100
    \item End of the Semester Project: 35/100
    \item Participation: 30/100

\end{itemize}

\section*{Lecture Plans}
Each lecture will help to answer a predefined question, and give the students a chance to think about their own opinion regarding the topic. 

\section*{Lecture Schedule}

\begin{table}[ht]

\begin{tabular}{|l|l|}
\hline
Date & Topic \\
\hline
1/26& What exactly is a mental illness?
 & Natasha \\
\hline
2/1 & What are the two most common mental health conditions?
 & Sophie\\
 \hline
2/8 & What can you do to help? & Michelle \\
\hline
2/15 & What are some common symptoms of mental illnesses & Natasha \\
\hline
2/22 & Why are mental illnesses so commonly overlooked? & Sophie \\
\hline
3/1 & Why are mental illnesses still considered a taboo in certain societies?
 & Michelle \\
\hline
3/8 & Why do people resort to suicide? & Natasha \\
\hline
3/15 & How do events such as walks and fundraisers help create awareness for the illnesses themselves and for the patients as well? 
 & Sophie \\
\hline
3/22 &  What is the difference between anxiety and depression?
 & Michelle \\
\hline
3/29 & How can stress eventually lead to a mental illness? & Natasha \\
\hline
4/5 & How has cognitive therapy evolved through the years?
 & Sophie \\
\hline
4/12 & More information about clinical depression? & Michelle \\
\hline
4/19 & Why have the cases of depression increased over the years? & Natasha \\
\hline
4/26 &  How to deal with common cases of mental illnesses like ADHD and OCD? & Sophie \\

\hline
\end{tabular}
\end{table}

Emails: natashabatra@berkeley.edu/ syoo06@berkeley.edu/ michellevo404793@berkeley.edu

\end{document}